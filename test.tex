\PassOptionsToPackage{unicode=true}{hyperref} % options for packages loaded elsewhere
\PassOptionsToPackage{hyphens}{url}
%
\documentclass[]{article}
\usepackage{lmodern}
\usepackage{amssymb,amsmath}
\usepackage{ifxetex,ifluatex}
\usepackage{fixltx2e} % provides \textsubscript
\ifnum 0\ifxetex 1\fi\ifluatex 1\fi=0 % if pdftex
  \usepackage[T1]{fontenc}
  \usepackage[utf8]{inputenc}
  \usepackage{textcomp} % provides euro and other symbols
\else % if luatex or xelatex
  \usepackage{unicode-math}
  \defaultfontfeatures{Ligatures=TeX,Scale=MatchLowercase}
\fi
% use upquote if available, for straight quotes in verbatim environments
\IfFileExists{upquote.sty}{\usepackage{upquote}}{}
% use microtype if available
\IfFileExists{microtype.sty}{%
\usepackage[]{microtype}
\UseMicrotypeSet[protrusion]{basicmath} % disable protrusion for tt fonts
}{}
\IfFileExists{parskip.sty}{%
\usepackage{parskip}
}{% else
\setlength{\parindent}{0pt}
\setlength{\parskip}{6pt plus 2pt minus 1pt}
}
\usepackage{hyperref}
\hypersetup{
            pdfborder={0 0 0},
            breaklinks=true}
\urlstyle{same}  % don't use monospace font for urls
\setlength{\emergencystretch}{3em}  % prevent overfull lines
\providecommand{\tightlist}{%
  \setlength{\itemsep}{0pt}\setlength{\parskip}{0pt}}
\setcounter{secnumdepth}{0}
% Redefines (sub)paragraphs to behave more like sections
\ifx\paragraph\undefined\else
\let\oldparagraph\paragraph
\renewcommand{\paragraph}[1]{\oldparagraph{#1}\mbox{}}
\fi
\ifx\subparagraph\undefined\else
\let\oldsubparagraph\subparagraph
\renewcommand{\subparagraph}[1]{\oldsubparagraph{#1}\mbox{}}
\fi

% set default figure placement to htbp
\makeatletter
\def\fps@figure{htbp}
\makeatother


\date{}

\begin{document}

\hypertarget{other-sql-clients}{%
\subsection{Other SQL Clients}\label{other-sql-clients}}

Heidi is only one out of many possible clients. We recommend it as a
good place to start if you have never used SQL before, but there are
also other options that are more suited to particular tasks. This list
is a few that the Stats4SD has some experience with that might be of
interest.

\hypertarget{table-plus}{%
\subsubsection{Table Plus}\label{table-plus}}

(download here){[}https://tableplus.io/windows{]}

This is a relatively new application - the Mac version was released in
2017 and the Windows version was released in Summer 2018.

The full version costs \$49 for a perpetual licence. There is a
free-forever version, and the only restriction I can find is that you
can only have 2 tabs open at once, limiting the number of things you can
do at once.

The company have a `getting started guide', that provides instructions
on performing some common tasks, such as creating a connection to a
database, viewsing and editing data, running queries, and importing and
exporting data from CSV files.

\textbf{positive} - The interface is clear and clean. It's easy to see
your tables, views and data. - The SQL code editor is very nice, with
autocompletion for database and object names. - The Free version has few
limitations compared to the paid version. The main limitation is in the
number of tabs you can have open at a time. (The free version is limited
to 2 tabs).

\textbf{negative} - As it's still new, it's missing some features that
other SQL clients have. - It has no visual query or relationship builder
(but then, neither does Heidi). - There is no easy way to visualise your
data structures - you can view each table individually, but not all
together.

\begin{itemize}
\tightlist
\item
  \textbf{Editor's Note} I personally use Table Plus for most of my SQL
  database work. I find its clean aesthetic is great for keeping all my
  databases organised (both local and ones on remote servers). It works
  for me because I also use the command line interface for doing more
  advanced workflows, like running database backups and moving databases
  between servers - so I don't miss the fact that these actions are hard
  or impossible within the software.
\end{itemize}

\hypertarget{for-anyone-learning-the-basics-of-how-databases-work-i-recommend-sticking-with-heidi-and-possibly-using-dbforge-to-help-you-visualise-the-relationshipss-in-your-structure-as-those-tools-are-more-fully-featured-and-better-supported.-but-for-people-who-are-more-comfortable-with-sql-i-recommend-trying-table-plus-to-see-if-it-works-for-you.}{%
\subsection{For anyone learning the basics of how databases work, I
recommend sticking with Heidi, and possibly using DBForge to help you
visualise the relationshipss in your structure, as those tools are more
fully featured (and better supported!). But for people who are more
comfortable with SQL, I recommend trying Table Plus to see if it works
for
you.}\label{for-anyone-learning-the-basics-of-how-databases-work-i-recommend-sticking-with-heidi-and-possibly-using-dbforge-to-help-you-visualise-the-relationshipss-in-your-structure-as-those-tools-are-more-fully-featured-and-better-supported.-but-for-people-who-are-more-comfortable-with-sql-i-recommend-trying-table-plus-to-see-if-it-works-for-you.}}

\hypertarget{dbforge}{%
\subsubsection{DBForge}\label{dbforge}}

(download
here){[}https://www.devart.com/dbforge/mysql/studio/download.html{]} -
Express is the free version of the software. Other versions offer a
trial version, which expires after 30 days.

DBForge is made of a set of different tools for interacting with your
databases. From the start screen, there are tabs that show you these
tools. Two features make DBForge a very useful tool to explore - the
Query Builder (under SQL Development tab) and the Database Diagram
(under Database Design tab).

\textbf{positive}

The Query builder is useful for people new to the SQL language, as it
allows you to build select queries, including complex features like
joins, through a visual interface.

The Database design tool is s

\end{document}
